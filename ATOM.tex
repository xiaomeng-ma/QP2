\section{Study 2. ATOM Model}
\subsection{ATOM Model}
Researchers have provided a wide range of explanations for pronoun case errors. From a theoretical perspective, pronoun case errors occur when the children fail to check both tense and agreement in their utterances according the to Agreement/Tense Omission Model (\textsc{atom}) \citep{wexler1998very,wexler1998subject}. They argue that children clearly know the case system at an early age. More specifically, they understand that the presence of agreement in INFL requires the NOM case to be asssigned to the subject. However, children experience an Optional Infinitive stage \citep{wexler199414}, during which children produce utterances omitting inflections and with fully inflected forms at the same time. Although the theory didn't further explain why would children knowingly omit some inflections at this stage, they do believe that the errors in pronoun case in children's utterances don't reflect their impaired knowledge of the case system. They observations of the data:
\begin{exe}
\ex Non-NOM subject errors > NOM-ACC object errors 
\ex NOM subjects all with INFL, Non-NOM subjects - some have INFL some don'
\end{exe}

\subsection{Methods}
In \cite{schutze1996subject}, they used data of three children (Nina, Peter and Sarah) in CHILDES. They included the all the files that have non-NOM subject, which results in 31 files for Nina (1;11 - 2;5), 9 files for Peter (1;11 - 2;5) and 21 files for Sarah (2;8 - 3;1). They only counted finite and non-finite INFL forms occurring with the pronoun, thus ignoring utterances like \textit{I go} or \textit{me go}. In their study, finite verb types include auxiliary, modal, copula and past tense; null auxiliary and null copula are considered INFL. They also excluded full or partial imitations, self-repetitions, clearly rote or formulaic expressions and instances where the utterances plus the situation did not make it clear what the intended subject was. 
In the early cases of all the three children, they all have utterances demonstrating that they know that the accusative case is the default case. For example, all three children used the accusative case in the elliptical answers such as (\ref{elip}), and none of the nominative case has appeared in such environment.
\begin{exe}
\ex \label{elip} Mother: Who's going to eat with a big spoon?\\
Nina: Me!
\end{exe}
They counted all the finite verbs and non-finite verbs occurring with first person singular pronouns (\textit{I, me, my}) and third person singular pronouns (\textit{he, she, him, her}). The results of their study is shown in table\ref{tab:ATOMSchutze}. 
\begin{table}[!]
    \caption{\cite{schutze1996subject}'s data on Nina, Peter and Sarah}
    \label{tab:ATOMSchutze}
   \begin{minipage}[t]{0.5\textwidth}
    \centering
    \subcaption{Nina 1sg Finiteness VS Case}
    \begin{tabular}{@{}lll@{}}
        \toprule
         & \multicolumn{2}{c}{Verb form}\\
         \cline{2-3}
        Subject & Finite & -Finite \\
        \midrule
        I & 40 & 45 \\
        me + my & 2 & 13 \\
        \hline
        \%non-NOM & 5\% & 22\% \\
        \bottomrule
    \end{tabular}
\end{minipage}
\vspace{1ex}
\begin{minipage}[t]{0.5\textwidth}
    \centering
    \subcaption{Nina 3sg Finiteness VS Case}
    \begin{tabular}{@{}lll@{}}
        \toprule
         & \multicolumn{2}{c}{Verb form}\\
         \cline{2-3}
        Subject & Finite & -Finite \\
        \midrule
        he + she & 255 & 139 \\
        him + her & 14 & 120 \\
        \hline
        \%non-NOM & 5\% & 46\% \\
    \bottomrule
    \end{tabular}
    \end{minipage}
\vspace{1ex}
    %\caption{Nina Distribution by Verb Type}
    \begin{minipage}[t]{0.5\textwidth}
    \centering
    \subcaption{Nina 1sg Distribution by Verb Type}
    \begin{tabular}{@{}llll@{}}
        \toprule
            &\multicolumn{3}{c}{Subject Form}\\
            \cline{2-4}
        Verb form & I & me & my \\
        \midrule
        Auxiliary & 0 & 0 & 0 \\
        Modal & 28 & 0 & 0 \\
        Copula & 0 & 0 & 0 \\
        Past Tense & 12 & 0 & 2 \\
        \hline
        Null Auxiliary & 36 & 0 & 8 \\
        Null Copula & 9 & 2 & 3 \\
        \bottomrule
    \end{tabular}
\end{minipage}
\vspace{1ex}
\begin{minipage}[t]{0.5\textwidth}
    \centering
    \subcaption{Nina 3sg Distribution by Verb Type}
    \begin{tabular}{@{}lllll@{}}
        \toprule
            &\multicolumn{4}{c}{Subject Form}\\
            \cline{2-5}
        Verb form & he & him & she & her \\
        \midrule
        Main V with -s & 9 & 0 & 1 & 0 \\
        Aux with -s & 120 & 1 & 5 & 0 \\
        Modal & 9 & 0 & 0 & 3 \\
        Copula with -s & 90 & 2 & 5 & 2 \\
        Past Tense & 15 & 0 & 1 & 6 \\
        \hline
        Main V without -s & 85 & 9 & 5 & 51 \\
        Aux with -s & 19 & 0 & 1 & 11 \\
        Null Auxiliary & 17 & 1 & 0 & 24 \\
        Null Copula & 12 & 0 & 0 & 24 \\
        \bottomrule
    \end{tabular}
\end{minipage}
\vspace{1ex}
   \begin{minipage}[t]{0.5\textwidth}
    \centering
    \subcaption{Peter 1sg Finiteness VS Case}
    \begin{tabular}{@{}lll@{}}
        \toprule
         & \multicolumn{2}{c}{Verb form}\\
         \cline{2-3}
        Subject & Finite & -Finite \\
        \midrule
        I & 243 & 29 \\
        me + my & 3 & 8 \\
        \hline
        \%non-NOM & 1.2\% & 22\% \\
        \bottomrule
    \end{tabular}
\end{minipage}
\vspace{1ex}
   \begin{minipage}[t]{0.5\textwidth}
    \centering
    \subcaption{Peter 1sg Distribution by Verb Type}
    \begin{tabular}{@{}llll@{}}
        \toprule
            &\multicolumn{3}{c}{Subject Form}\\
            \cline{2-4}
        Verb form & I & me & my \\
        \midrule
        Auxiliary & 110 & 0 & 0 \\
        Modal & 54 & 0 & 0 \\
        Copula & 10 & 0 & 0 \\
        Past Tense & 69 & 2 & 1 \\
        \hline
        Null Auxiliary & 29 & 4 & 2 \\
        Null Copula & 0 & 1 & 1 \\
        \bottomrule
    \end{tabular}
\end{minipage}
\textbf{\linebreak}
\begin{minipage}[t]{0.5\textwidth}
    \centering
    \subcaption{Sarah 3sg Finiteness VS Case}
    \begin{tabular}{@{}lll@{}}
        \toprule
         & \multicolumn{2}{c}{Verb form}\\
         \cline{2-3}
        Subject & Finite & -Finite \\
        \midrule
        she & 21 & 24 \\
        her & 3 & 14 \\
        \hline
        \%non-NOM & 13\% & 37\% \\
    \bottomrule
    \end{tabular}
    \end{minipage}
\begin{minipage}[t]{0.5\textwidth}
    \centering
    \subcaption{Sarah 3sg Distribution by Verb Type}
    \begin{tabular}{@{}lll@{}}
        \toprule
            &\multicolumn{2}{c}{Subject Form}\\
            \cline{2-3}
        Verb form & she & her \\
        \midrule
        Main V with -s & 3 & 0 \\
        Aux with -s & 7 & 0 \\
        Modal & 3 & 0 \\
        Copula with -s & 1 & 0 \\
        Past Tense & 7 & 3 \\
        \hline
        Main V without -s & 9 & 10  \\
        Aux with -s & 1 & 0  \\
        Null Auxiliary & 7 & 3 \\
        Null Copula & 7 & 1  \\
        \bottomrule
    \end{tabular}
\end{minipage}
\end{table}

As shown in the table, for Nina, Peter and Sarah, theiir non-nominitive case pronouns occur with non-finite verbs more frequent than finite verbs, which provides evidence for the ATOM hypothesis. 


\subsection{My methodology}
In order to, \%gra tier is used to search for the verbs. Untensed verbs are tagged as \textit{'v'} or \textit{'v-root'}. Verbs with INFL are tagged detailely in the \% tier. Next table showed a breakdown of the verb types following each pronoun. 
Auxililaries include any tags that include \textit{'aux'}. Modals include any tags that include \textit{'mod}. Copulas include any tags that include \textit{'cop'}. Past tense verbs include verbs that have been tagged with \textit{'-PAST'}. Null Auxiliary include negation marker such as \textit{'not'}, which is tagged as \textit{'neg'}; progressive and perfective forms of the verb, which is tagged as \textit{'part'}. Null Copula include adjectives, adverbs and prepositions %\footnote{The adverb \textit{'too'} is not counted.}. Nouns are not included in Null Copula because Nouns are more suitable to count for GEN errors. 
My data:
This is all the output from the bloody latex. 
    




\section{Introduction}
A common claim in the first language acquisition literature is that English-speaking children make pronoun case errors from the age of two to four. The most common three types of error are: (a) using an accusative form in the nominative context, (b) using a nominative form in the accusative context, and (c) using a genitive form in the nominative context, as exemplified in examples (1) to (3) respectively. 

\begin{exe}
\ex [*] {when me fall in the swimming pool. - \textit{me} for \textit{I} (Nina, 2;5 : \citet{suppes1974semantics} )} 
\ex [*] {Mama put he here. - \textit{he} for \textit{him} (Tow, 1;9 : \citet{post1993language})} 
\ex [*] {My comb hair. - \textit{my} for \textit{I} (Adam, 2;3 : \cite{brown1973first} )} 
\end{exe}

These errors are unlikely to be due to confusion about which pronoun to use in what context, as children usually produce the correct and incorrect forms at the same time. Moreover, some pronouns, such as \textit{me/I} and \textit{her/she}, are more vulnerable to case errors than other pronouns \citep[e.g.][]{schutze1996subject,rispoli1998,rispoli1999,pine2005testing}. There is individual variation by child as well. Some children never produce any pronoun case errors; whereas some make errors on one particular pronoun, and some produce all types of errors \citep{rispoli2005}.  




The \textsc{atom} successfully explains the pattern in which the verb accompanying an incorrect pronoun is usually untensed; however, it fails to account for the fact that different pronouns yielf different numbers of errors. Taking children's production into consideration,  \cite{rispoli1998,rispoli1999,rispoli2005} has proposed a  \textsc{paradigm building} model:for each pronoun, case, person, and number form a 3x3 paradigm. Young children have difficulties accessing all forms of pronouns in the paradigm, and thus make pronoun errors. The \textsc{paradigm building} model explains variability in children's pronoun error production since lexical retrieval of a certain pronoun could fail for various reasons. Many researchers suggest that parents' input should also play a part in pronoun case errors. The distribution patterns of the pronoun, such as \textit{Let \underline{me do} it} and \textit{\underline{Her drink} is over there}, can be misleading and result in children's pronoun case errors \citep[e.g.][]{budwig1996influences,tomasello2000,tomasello2003, ambridge2006testing,kirjavainen2009can}.


This paper will investigate pronoun case errors in children's production by conducting a comprehensive corpus analysis on all the available data of English-speaking children on \textsc{childes} \citep{macwhinney2014childes}. First, this paper will review how often does pronoun case errors occur in children's utterances. Although it is treated as a common error in children's speech, the frequency of pronoun case error varies largely from study to study, depending on how the data were collected (corpus data or experimental data) and how the data were coded. This paper will review the frequency of case errors reported in the previous studies and compare them to the error rate found in all the corpora used in this study in order to determine the nature and number of pronoun case errors.

Second, this paper will examine the \textsc{atom} model, the \textsc{paradigm building} model and the input-driven hypothesis using the same set data. In the previous studies, researchers either collected their own data or used different corpora to test their hypotheses. It is necessary to evaluate different theories on the same set of data in order to control for collection bias or limitations of small sample size. 

Third, this paper will provide a hypothesis about how children acquire pronoun case in the first place. Cases are used to mark different relationship between arguments, which is a more abstract grammatical feature to learn compared to plural forms or tense marking. Yet young children are able to produce the correct form of a pronoun in different argument positions. It would be worthwhile to ask the question: how do children acquire this abstract feature at such a young age? This study is going to test two hypotheses of the acquisition of pronoun case. From a theoretical perspective, the children could derive different cases through argument structure. Nominative case and accusative case could be differentiated at the sentential level, such that the former is associated with the subject and the latter case is associated with the object. In addition, children could also acquire cases through statistical learning. Nominative case, accusative case and genitive case have different distributional patterns in speech, which difference could be utilized by children to acquire different cases. 

\subsection{What I Argue}
Children can use case in a very early age. However, there is not enough evidence to show that they understand that nominative case is assigned by the INFL feature of the verb. Instead, this paper is going to argue, children's early understanding of the case reflects their statistical learning ability on the distribution of different pronouns. 
\section{Study 1. General Data Description}
\subsection{Corpora}
To avoid the limitation of small sample size and selection bias, this study will conduct a comprehensive corpus analysis of speech from monolingual English-speaking children from roughly ages two to four. That includes 46 children with longitudinal recordings and 211 children with cross-sectional recordings (see Table 1 and Table 2).
\subsection{Data Coding}
This study will apply the \textsc{nltk} python package to automatically extract data from the annotated corpora in CHILDES. The data that has been collected from each child's file include their age, mlu, total number of words and input words, total number of sentences and input sentences, total number of each pronoun and each input pronoun. All of these data can be subtracted directly from the corpus using \textsc{nltk} python package.
\subsection{Error Coding}
To identify pronoun errrors in each file, the part-of-speech taggers in CHILDES is used. CHILDES uses MOR and GRASP programs to annotate part-of-speech taggers and dependency grammatical relations for all transcripts. The MOR program produces the \textit{\%mor} tier, in which part-of-speech tags stand in one-to-one correspondence with the word in the transcript line. The GRASP program produces the \textit{\%gra} tier which represents grammatical relations. For English data, the automated annotation system has been reported to have high-level accuracy: the MOR program reaches 97\% of accuracy and the precision for determining subject and object relation is reportedly 95.8\% and 94.1\% for GRASP output \citep{macwhinney2012morphosyntactic,sagae2010morphosyntactic}.  Below is an excerpt from Eve in Brown corpus that exemplifies the \textit{\%mor} tier and \textit{\%gra} tier.
\begin{exe}
\ex \label{lookathe}\gll *CHI:	look at he .\\
\%mor:	v|look prep|at pro:sub|he .\\
\%gra:	1|0|ROOT 2|1|JCT 3|2|POBJ 4|1|PUNCT\\
(Fraser/020604b.xml)
\ex \label{meseehe}\gll *CHI: when me see he again ?\\
\%mor: conj|when pro:obj|me v|see pro:sub|he adv|again ? \\
\%gra: 1|3|LINK 2|3|SUBJ 3|0|ROOT 4|3|OBJ 5|4|JCT 6|3|PUNCT \\
(Brown/Eve/020100.xml)

\ex \label{myfine} \gll *CHI:	my find paper .\\
	\%mor:	det:poss|my v|find n|paper .\\
	\%gra:	1|2|SUBJ 2|0|ROOT 3|2|OBJ 4|2|PUNCT\\
(Brown/Adam/020304.xml)
\end{exe}
\subsubsection{NOM case error and ACC case error}
The nominative case errors and accusative case errors are identified on \textit{\%gra} tier. Since nominative case and accusative case are assigned at a sentential level, \textit{\%gra} tier is more appropriate to use since it represents grammatical dependency relations. For nominative case pronouns (\textit{I, he, she, we, they}), if they appear in a sentence with dependency grammatical relations \footnote{Children produce many incomplete sentences and run-on sentences. Not all the incomplete sentences have dependency grammatical relations, such as \textit{'I yeah'} and \textit{'um I I I.'}. To have dependency grammatical relations, the sentence needs to have at least two words from different grammatical categories that can be combined together. Sentences without dependency grammatical relations are excluded in this study.}if the pronoun is not tagged as \textit{SUBJ} on the \textit{\%gra} tier, it is counted as an error. For example, in sentence (\ref{lookathe}), nominative case pronoun \textit{'he'} is used in an object position and it is tagged as \textit{POBJ} (object) on the \textit{\%gra} tier. 



Similar to the searching process of nominative case pronoun errors, for accusative case pronouns \textit{me, him, us, them}, if they appear in a sentence with dependency grammatical relations and if they are not tagged as \textit{OBJ} or \textit{POBJ}, they are counted as error. For example, in sentence (\ref{meseehe}), \textit{me} is an accusative case pronoun and it was used at the subject position which was tagged as \textit{SUBJ}. The 3sg feminine pronoun \textit{her} is a special case, since it is used for genitive case. The \textit{\%mor} tier is used to distinguish accusative \textit{her} and genitive \textit{her}. The former is tagged as \textit{pro:obj} and the latter is tagged as \textit{det:poss}. For all the \textit{her}s that is tagged as \textit{pro:obj} on the \textit{\%mor} but not tagged as \textit{OBJ} or \textit{POBJ} on the \textit{\%gra} tier, they are counted as error. 
\subsection{GEN case error}
Genitive case errors were not identified with \textit{\%gra} tier, since the \textit{\%gra} tier is less successful in capturing phrasal relationship. The genitive case pronoun are  not always tagged as \textit{SUBJ} or \textit{OBJ} even though they appeared in the subject or object position. For example in (\ref{mywantit}), the genitive case pronoun \textit{my} is used at the subject position, but it was tagged as \texit{DET} (determiner) on the \texit{\%gra} tier. 
\begin{exe}
\ex \label{mywantit} \gll  *CHI:	my [*] want it .  \\     
\%mor:	det:poss|my v|want pro:per|it .\\
\%gra:	1|2|DET 2|0|ROOT 3|2|OBJ 4|2|PUNCT\\
(Eleanor/021004a.cha)
\end{exe}

Genitive case errors are identified using \textit{\%mor} tier by exhausting the possible part-of-speech combinations of determiners. 
\subsection{Exclusion Criterion}
Along with case errors, children also make many other types of errors. As long as the error can not be resolved by changing the case of the pronoun, those type of errors are not included. 
\begin{exe}
\ex \gll *CHI:	I dead him .\\
\%mor:	pro:sub|I adj|dead pro:obj|him .\\
\%gra:	1|3|SUBJ 2|3|MOD 3|0|INCROOT 4|3|PUNCT\\ 
(Fraser/020405a)
\end{exe}